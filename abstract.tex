\specialhead{ABSTRACT}

Image classification is an important problem. It is an approach of pattern recognition in computer vision. The goal of the approach is to differentiate between different classes of images based on the quantification of contextual information in the images. Image classification is performed using data analytic pipelines. These pipelines are organized as interdependent and individual components. These components include image acquisition, image preprocessing, feature extraction, feature preprocessing, dimensionality reduction, learning algorithms. More steps may be added or removed from the pipeline. The quality of the image classification pipeline is measured by the error in classification of new image instances. This error is an accumulation of the errors introduced from different parts of the pipeline starting from the acquisition of the image from the sample to the algorithm used for performing the  classification. This error is propagated down the pipeline which finally accumulated in the form of the aforementioned classification error. In this thesis, we attempt to reduce, quantify and understand the error in image classification pipelines.   

In the first part, we try to reduce the error in a holistic manner. The problem selected for demonstrating this is that of automatic microstructure recognition. The dataset for this problem is from the domain of material science. It consists of images of microstructures in the micrometer scale. We perform two classification tasks. The first task is of differentiating between dendritic and non-dendritic microstructures. The second task is for differentiating between longitudinal and transverse cross sections of microstructures. The objective for both the classification tasks is to find the best set of algorithms and hyper-parameters for minimizing the classification error. The approach used was an exhaustive search of the configuration space of algorithms and corresponding hyperparameters for three stages of the pipeline - feature extraction, dimensionality reduction and learning algorithms. This exhaustive grid search is able to perform classification of the two tasks with a sufficiently high degree of accuracy.


