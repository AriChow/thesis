%%% INTRODUCTION AND BACKGROUND
\chapter{INTRODUCTION AND BACKGROUND}
\label{chap:intro}

%%% INTRODUCTION
\section{Introduction}

Numerical simulation has become ubiquitous in engineering and scientific
practice due to the continuing increase in power and accessibility of
computational resources. For scientific and engineering applications, ensuring
the accuracy and reliability of the computed numerical solution is of primary
importance. For example, it is often necessary to design structural components
for which the von-Mises stress in the component's domain is less than a given
material yield strength when subjected to a variety of loading conditions.

In the context of finite element methods, \emph{a posteriori} error estimation
and mesh adaptation provide useful tools for approximating and controlling
the discretization error when solving partial differential equations
(PDEs) (\emph{c.f} \cite{ainsworth2011posteriori, gratsch2005posteriori,
verfurth1994adaptive}.)

In an abstract finite element setting, one seeks to solve the variational
problem: find $u \in \S$ such that
%
%% aut_abstract_variational
\begin{gather}
\R(w; u) = 0 \quad \forall w \in \V,
\end{gather}
%
where $\R: \S \times \V \to \mathbb{R}$ is a semilinear form, linear in its
first argument and potentially nonlinear in its second, and $\S$ and
$\V$ are Hilbert spaces. Given the exact solution $u$ to this problem and
a corresponding finite element approximation $u^H$, the discretization error
is defined as $e := u -u^H$. Traditional \emph{a posteriori} error estimates
attempt to bound or approximate the discretization error in a global norm
$\| e \|$, such as the energy norm induced by the underlying partial
differential operator. In the past twenty years, though, adjoint-based
techniques have been developed and utilized to obtain approximations $\eta$
or approximate bounds $\hat{\eta}$ for the error $|J(u) - J(u^H)|$ in some
physically meaningful functional $J : \V \to \mathbb{R}$, such that
%
%% aut_functional_bounds_abstract
\begin{gather}
|J(u) - J(u^H)| \approx \eta < \hat{\eta}.
\end{gather}

The use of adjoint (or duality) techniques for \emph{a posteriori} error
estimation can be traced back to Babu\v{s}ka and Miller
\cite{babuska1984post1, babuska1984post2, babuska1984post3} who explored the
post-processing of functional quantities. These ideas were then expaned to the
context of adaptive finite element methods by Johnson et al.
\cite{eriksson1996computational}. Becker and Rannacher \cite{becker2001optimal}
developed an approach to functional \emph{a posteriori} error estimation for
Galerkin finite element methods called the \emph{dual weighted residual}
method. Giles and Pierce \cite{giles2003adjoint} developed an approach
conceptually similar to the dual weighted residual method, but additionally
concerned themselves with discretizations that lack Galerkin orthogonality,
such as Godunov finite volume methods. Venditti and Darmofal
\cite{venditti2000adjoint, venditti2002adjoint, venditti2003adjoint} developed
adjoint-based error estimates for arbitrary discretizations using discrete
adjoint equations based on two-level discretization schemes.

Adjoint-based error estimation and mesh adaptation have been heavily adopted
by the computational fluid dynamics (CFD) community \cite{fidkowski2011review}.
This can in part be explained by the fact that the current increase in
computing power alone may not be sufficient to accurately resolve
quantities of interest (QoIs) in CFD applications, even when very finely
generated \emph{a priori} meshes are used \cite{fidkowski2011review}. However,
despite its popularity in CFD, adjoint-based error estimation is not regularly
applied to solid mechanics applications.

In the context of solid mechanics, adaptive adjoint-based error estimation
has been used to study linear elasticity in two
\cite{rannacher1997feed, stein2007error, gonzalez2014mesh} and three
\cite{ghorashi2014goal} dimensional elasticity, two
\cite{rannacher1998posteriori, rannacher1999posteriori} and three
\cite{ghorashi2017goal} dimensional elasto-plasticity, two dimensional
thermoelasticity \cite{rabizadeh2015adaptive}, two dimensional
nonlinear elasticity \cite{larsson2002strategies}, and two dimensional
hyperelasticity \cite{whiteley2014error}. We remark that in the majority of
this literature, mesh adaptation is performed with structured adaptive mesh
refinement using quadrilateral or hexahedral elements and without any
discussion of parallelization. Additionally, none of these investigations
apply the current state of the art capabilities of parallel three-dimensional
unstructured mesh adaptation, as is currently the norm for CFD
applications.

Perhaps one reason for this discrepancy is the fact that adjoint-based error
estimation requires the development and implementation of a number of
non-trivial steps. From a high-level, the following steps must be carried out
to perform an adaptive adjoint-based error analysis:
%
\begin{enumerate}
\item Solve the original (primal) PDE of interest.
\item Using the primal solution, derive and solve an auxiliary adjoint PDE.
\item Enrich the solution to the adjoint PDE in some manner.
\item Compute error estimates using an adjoint-weighted residual method.
\item Localize the error to contributions at the mesh entity level.
\item Adapt the mesh based on the local error contributions.
\end{enumerate}
%
Additionally, modern solid applications necessitate the use of parallel
analysis, meaning each of these steps must execute effectively and
efficiently on parallel machines.

Further, both the primal residual and functional QoI can be highly nonlinear
for solid mechanics applications. Solving both the primal and adjoint problems
requires the linearization of the primal residual and the functional QoI
with respect to the degrees of freedom of the problem. The derivation and
numerical implementation of these linearizations can be time consuming and
error-prone. As a result, adjoint-based error control has typically remained
a tool for expert analysis with a high barrier of entry.

As a final observation, we note that unstructured mesh generation and
adaptation is robust, reliable, and scalable for simplical elements. However,
for solid mechanics applications with incompressibility constraints, such as
isochoric plasticity or incompressible hyperelasticity, standard
displacement-based Galerkin finite element discretizations are known to be
unstable when using simplical elements. This fact may have further hindered
the adoption of unstructured mesh adaptation for previous adaptive
adjoint-based solid mechanics applications.

%%% OUTLINE
\section{Outline}

%%% CONTRIBUTIONS
\section{Contributions}

\begin{enumerate}
\item The development and implementation of an automated approach
for adjoint-based error that executes on parallel machines.
\item The application of the previously discussed automated approach
to conduct an investigation of adjoint-based error estimation and mesh
adaptation for finite deformation elasticity with stabilized finite
elements.
\item The application of the previously discussed automated approach
to conduct an initial investigation of a non-uniform refinement approach
for solving adjoint problems in the context of adjoint-based error
estimation.
\item The development of a novel approach for adjoint-based error
estimation in variational multiscale finite element methods.
\item A proof that two drasticallly distinct approaches for
adjoint-based error estimation approaches in variational multiscale
finite element methods yield identical results.
\end{enumerate}
