\chapter{PROPOSITIONS FOR THE ADVECTION-DIFFUSION OPERATOR}
\label{chap:vmsapp}

\let\thefootnote\relax\footnotetext{
This chapter previously appeared as:
B.~N. Granzow, M.~S. Shephard, and A.~A. Oberai,
``Output-based error estimation and mesh adaptation
for variational multiscale methods.''
\emph{Comput. Methods in Appl. Mechanics and Eng.},
vol. 322, pp. 331-459, Aug. 2017.}

%%%%%%%%%%%%%%%%%%%%%%%%%%%%%%%%%%%%%%%%%%%%%%%%%%%%%%%%%%%%%%%%%%%%%%%%%%%%%%
\section{Non-homogeneous boundary conditions}
\label{app:bcs}
%%%%%%%%%%%%%%%%%%%%%%%%%%%%%%%%%%%%%%%%%%%%%%%%%%%%%%%%%%%%%%%%%%%%%%%%%%%%%%

Extensions to non-homogeneous Dirichlet boundary conditions
for the primal model, given as
%
\begin{gather}
\begin{cases}
\begin{aligned}
\OP u &= f, & & \bs{x} \in \Omega, \\
u &= g, & & \bs{x} \in \partial \Omega,
\end{aligned}
\end{cases}
\label{eq:bcs_primal_pde}
\end{gather}
%
can readily be made by introducing the decomposition
$u = u_0 + \tilde{g}$, where $\text{tr}({\tilde{g}}) = g$.
The problem is then reposed as a homogeneous
Dirichlet problem given by
%
\begin{gather}
\begin{cases}
\begin{aligned}
\OP u_0 &= f + \OP \tilde{g}, & & \bs{x} \in \Omega, \\
u_0 &= 0, & & \bs{x} \in \partial \Omega,
\end{aligned}
\end{cases}
\end{gather}
%
where all arguments made previously can be applied to this
modified formulation, provided $f + \OP \tilde{g} \in \V^*$.

Extensions to non-homogeneous Neumann boundary conditions
require additional investigation. To proceed,
consider the primal problem given as
%
\begin{gather}
\begin{cases}
\begin{aligned}
\OP u &= f & & \bs{x} \in \Omega, \\
u &= 0, & & \bs{x} \in \partial \Omega_D \\
B u &= h & & \bs{x} \in \partial \Omega_N,
\end{aligned}
\end{cases}
\label{eq:dbcs_primal_pde}
\end{gather}
%
where $\Omega_D \cup \Omega_N = \Omega$ and $\Omega_D \cap \Omega_N =
\{ \varnothing \}$. Multiplying the left hand side of the
primal problem \eqref{eq:dbcs_primal_pde}
by an arbitrary test function $v$ and integrating by parts over the
domain twice yields the relationship
%
\begin{gather}
\int_{\Omega} v \OP u \, \text{d} \Omega +
\int_{\partial \Omega_N} v B u \, \text{d} \Gamma =
\int_{\Omega} \OP^* v u \, \text{d} \Omega +
\int_{\partial \Omega_N} B^* v u \, \text{d} \Gamma.
\end{gather}
%
All subsequent derivations would need to be
made considering this relationship, which involves
the boundary operator $B$, rather than
relationship \eqref{eq:adjoint} which has
been used extensively in this paper.

%%%%%%%%%%%%%%%%%%%%%%%%%%%%%%%%%%%%%%%%%%%%%%%%%%%%%%%%%%%%%%%%%%%%%%%%%%%%%%
\section{Derivation of the advection-diffusion adjoint operator}
\label{app:adjoint}
%%%%%%%%%%%%%%%%%%%%%%%%%%%%%%%%%%%%%%%%%%%%%%%%%%%%%%%%%%%%%%%%%%%%%%%%%%%%%%

Let $\OP: \V \to \V^*$
be the steady-state, constant coefficient operator
utilized in section \ref{sec:Results}:
%
\begin{gather}
\OP u := \adv{u}
\end{gather}
such that $\V = H^1_0(\Omega)$ and $\V^* = H^{-1}(\Omega)$.
To determine the corresponding operator:
$\OP^*$ that satisfies the adjoint property:
%
\begin{gather}
\paira{z}{\OP u} = \pairb{\OP^* z}{u}
\quad \forall \, u, z \in H^1_0(\Omega),
\end{gather}
%
we multiply $\OP u$ by an
arbitrary function $z \in H^1_0(\Omega)$ and
repeatedly apply the divergence theorem.
This proceeds as follows:
%
\begin{align*}
\pairb{z}{\OP u}
&=
\int_{\Omega} z (\adv{u}) \, \text{d} \Omega \\
&=
- \int_{\Omega} z \kappa \nabla^2 u \, \text{d} \Omega +
\int_{\Omega} z \bs{a} \cdot \nabla u \, \text{d} \Omega \\
&=
-\int_{\partial \Omega} z \kappa \nabla u \cdot \bs{n} \, \text{d} \Gamma +
\int_{\Omega} \kappa \nabla z \cdot \nabla u \, \text{d} \Omega + \\
& \quad \quad \quad \int_{\partial \Omega} (z \bs{a} u) \cdot \bs{n} \, \text{d} \Gamma -
\int_{\Omega} \bs{a} \cdot \nabla z u \, \text{d} \Omega \\
&=
\int_{\Omega} \kappa \nabla z \cdot \nabla u \, \text{d} \Omega -
\int_{\Omega} \bs{a} \cdot \nabla z u \, \text{d} \Omega \\
&=
\int_{\partial \Omega} (\kappa \nabla z u) \cdot \bs{n} \, \text{d} \Gamma -
\int_{\Omega} \kappa \nabla^2 z u \, \text{d} \Omega -
\int_{\Omega} \bs{a} \cdot \nabla z u \, \text{d} \Omega \\
&=
- \int_{\Omega} \kappa \nabla^2 z u \, \text{d} \Omega -
\int_{\Omega} \bs{a} \cdot \nabla z u \, \text{d} \Omega \\
&=
\int_{\Omega} (\adva{z}) u \,
\text{d} \Omega \\
&=
\paira{\OP^* z}{u}
\end{align*}
%
Here the third equality is achieved by application
of the divergence theorem to both terms,
the fourth equality holds since $z \in H^1_0(\Omega)$,
the fifth equality is achieved by application of the
divergence theorem to the leftmost term, and the sixth
equality holds since $u \in H^1_0(\Omega)$. Thus, the
operator $\OP^* : H^1_0(\Omega) \to H^{-1}(\Omega)$ is defined as
%
\begin{gather}
\OP^* z := \adva{z}.
\end{gather}
%
We make the observation that there has been a sign
change for the advective term since the operator
$\OP$ is not self-adjoint.
This sign change, however, is absorbed in the definition
of the operator $\OP^*$ and in no way introduces
a sign change in the fundamental property:
%
\begin{gather}
\pairb{z}{\OP u} = \paira{\OP^* z}{u} \quad \forall \, u, z \in H^1_0(\Omega).
\label{eq:adv_adjoint}
\end{gather}

%%%%%%%%%%%%%%%%%%%%%%%%%%%%%%%%%%%%%%%%%%%%%%%%%%%%%%%%%%%%%%%%%%%%%%%%%%%%%%
\section{Propositions applied to the advection-diffusion operator}
\label{app:propositions}
%%%%%%%%%%%%%%%%%%%%%%%%%%%%%%%%%%%%%%%%%%%%%%%%%%%%%%%%%%%%%%%%%%%%%%%%%%%%%%

We restate the adjoint property \eqref{eq:adv_adjoint} as
%
\begin{gather}
\int_{\Omega} z(\adv{u}) \, \text{d} \Omega =
\int_{\Omega} (\adva{z}) u \, \text{d} \Omega
\quad \forall \, u,z \in H^1_0(\Omega)
\label{eq:app_adjoint}
\end{gather}
%

We now define the primal problem as:
%
\begin{gather}
\begin{cases}
\begin{aligned}
\adv{u} &= f, & & \bs{x} \in \Omega, \\
u &= 0, & & \bs{x} \in \partial \Omega,
\end{aligned}
\end{cases}
\label{eq:app_primal_pde}
\end{gather}
%
corresponding to equation \eqref{eq:primal_pde},
where $f \in H^{-1}(\Omega)$.
We note that the primal residual operator is given as:
%
\begin{gather}
\R u := f + \madv{u}.
\label{eq:app_primal_residual}
\end{gather}
We define the continuous variational multiscale formulation
of the primal problem as: find $\bar{u} \in \bar{\V}$ such that
%
\begin{gather}
\int_{\Omega} (\adva{\bar{v}}) \G' \R \bar{u} \, \text{d} \Omega =
\int_{\Omega} \bar{v} \R \bar{u} \, \text{d} \Omega
\quad \forall \, \bar{v} \in H^1_0(\Omega),
\label{eq:app_primal_vms}
\end{gather}
%
corresponding to equation \eqref{eq:primal_vms_2}, where we
leave the fine-scale Green's operator
$\G' : H^{-1}(\Omega) \to H^1_0(\Omega)$
as an unspecified
abstract operator.
Here we note that $\G' \R \bar{u} \in H^1_0(\Omega)$.
Let $\V^h \subset \V$ denote a classical finite element space consisting
of piecewise linear functions defined over a discretization of the
domain $\Omega$. The primal subgrid model can then be stated as: find
$u^h \in \V^h$ such that
%
\begin{gather}
\sum_{e=1}^{n_{el}} \int_{\Omega^e} (\adva{v^h}) ( \uptau^e \R u^h)
\, \text{d} \Omega =
\int_{\Omega} v^h \R u^h \, \text{d} \Omega
\quad \forall \, v^h \in \V^h,
\label{eq:app_primal_subgrid}
\end{gather}
%
corresponding to equation \eqref{eq:primal_subgrid_2}, where we
leave $\uptau^e$ unspecified.

We define the dual problem as:
%
\begin{gather}
\begin{cases}
\begin{aligned}
\adva{z} &= q, & & \bs{x} \in \Omega, \\
z &= 0, & & \bs{x} \in \partial \Omega,
\end{aligned}
\end{cases}
\label{eq:app_dual_pde}
\end{gather}
%
corresponding to equation \eqref{eq:dual_pde},
where $q \in H^{-1}(\Omega)$. We note that the dual residual
operator is given as:
%
\begin{gather}
\R^* z := q + \madva{z}.
\label{eq:app_dual_residual}
\end{gather}
%
We define the continuous variational multiscale formulation
of the dual problem as: find $\bar{z} \in \bar{\V}$ such that
%
\begin{gather}
\int_{\Omega} \G'_d \R^* \bar{z} (\adv{\bar{v}}) \, \text{d} \Omega =
\int_{\Omega} \R^* \bar{z} \bar{v} \, \text{d} \Omega
\quad \forall \, \bar{v} \in H^1_0(\Omega),
\label{eq:app_dual_vms}
\end{gather}
corresponding to equation \eqref{eq:dual_vms_2}, where again
we leave the dual fine-scale Green's operator
$\G'_d : H^{-1}(\Omega) \to H^1_0(\Omega)$ unspecified.
We note that $\G'_d \R^* \bar{z} \in H^1_0(\Omega)$
The dual subgrid model can then be stated as: find
$z^h \in \V^h$ such that
%
\begin{gather}
\sum_{e=1}^{n_{el}} \int_{\Omega^e} (\uptau^e_d \R^* z^h) (\adv{v^h})
\, \text{d} \Omega =
\int_{\Omega} \R^* z^h v^h \, \text{d} \Omega
\quad \forall \, v^h \in \V^h,
\label{eq:app_dual_subgrid}
\end{gather}
%
corresponding to equation \eqref{eq:dual_subgrid_2}, where we leave
$\uptau^e_d$ unspecified.

%=============================================================================
\subsection{Proposition 2}
%=============================================================================
For any solutions $u = u' + \bar{u}$ to the continuous VMS
formulation \eqref{eq:app_primal_vms} and $z = z' + \bar{z}$ to the
continuous dual VMS formulation \eqref{eq:app_dual_vms}, we
derive the error representation:
%
\begin{gather*}
\begin{aligned}
J(u) - J(\bar{u})
&=
\int_{\Omega} q u \, \text{d} \Omega -
\int_{\Omega} q \bar{u} \, \text{d} \Omega \\
&=
\int_{\Omega} (\adva{z}) u \, \text{d} \Omega -
\int_{\Omega} (\adva{z}) \bar{u} \, \text{d} \Omega \\
&=
\int_{\Omega} z (\adv{u}) \, \text{d} \Omega -
\int_{\Omega} z (\adv{\bar{u}}) \, \text{d} \Omega \\
&=
\int_{\Omega} z f \, \text{d} \Omega -
\int_{\Omega} z (\adv{\bar{u}}) \, \text{d} \Omega \\
&=
\int_{\Omega} z \R \bar{u} \, \text{d} \Omega \\
&=
\int_{\Omega} z' \R \bar{u} \, \text{d} \Omega +
\int_{\Omega} \bar{z} \R \bar{u} \, \text{d} \Omega \\
&=
\int_{\Omega} z' \R \bar{u} \, \text{d} \Omega +
\int_{\Omega} (\adva{\bar{z}}) \G' \R \bar{u} \, \text{d} \Omega \\
&=
\int_{\Omega} (\G'_d \R^* \bar{z}) \R \bar{u} \, \text{d} \Omega +
\int_{\Omega} (\adva{\bar{z}}) \G' \R \bar{u} \, \text{d} \Omega \\
&=
\pairb{\G'_d \R^* \bar{z}}{\R \bar{u}} +
\paira{\OP^* \bar{z}}{\G' \R \bar{u}}
\end{aligned}
\label{eq:app_error_cont_2}
\end{gather*}
Here the first equality is by definition \eqref{eq:functional},
the second equality is due to the dual PDE \eqref{eq:app_dual_pde},
the third equality is due to the fundamental relation
\eqref{eq:app_adjoint}, the fourth equality is due to the primal
PDE \eqref{eq:app_primal_pde}, the fifth equality is due to the
definition of the primal residual \eqref{eq:app_primal_residual},
the sixth equality is due to the sum decomposition of the
dual solution $z = z' + \bar{z}$, the seventh equality is due
to the continuous variational formulation of the primal problem
\eqref{eq:app_primal_vms}, the eight equality is due to
the definition of the fine-scale dual solution
\eqref{eq:dual_fine_operator}, and the ninth equality
is due to the definition of the duality pairing we have chosen.

%=============================================================================
\subsection{Proposition 4}
%=============================================================================

For any solutions $u$ to the primal model \eqref{eq:primal_weak},
$z$ to the dual model \eqref{eq:dual_weak}, $u^h$ to the primal
subgrid model \eqref{eq:primal_subgrid} and $z^h$ to the dual
subgrid model \eqref{eq:dual_subgrid}, we derive the error
representation
%
\begin{gather*}
\begin{aligned}
J(u) - J(u^h)
&=
\int_{\Omega} q u \, \text{d} \Omega -
\int_{\Omega} q u^h \, \text{d} \Omega \\
&=
\int_{\Omega} (\adva{z}) u \, \text{d} \Omega -
\int_{\Omega} (\adva{z}) u^h \, \text{d} \Omega \\
& =
\int_{\Omega} z (\adv{u}) \, \text{d} \Omega -
\int_{\Omega} z (\adv{u^h}) \, \text{d} \Omega \\
&=
\int_{\Omega} z f \, \text{d} \Omega -
\int_{\Omega} z (\adv{u^h}) \, \text{d} \Omega \\
&=
\int_{\Omega} z \R u^h \, \text{d} \Omega \\
&=
\int_{\Omega} z \R u^h \, \text{d} \Omega -
\int_{\Omega} z^h \R u^h \, \text{d} \Omega + \\
& \quad \quad \quad \sum_{e=1}^{n_{el}} \int_{\Omega} (\adva{z^h})(\uptau^e \R u^h)
\, \text{d} \Omega \\
&=
\int_{\Omega} (z - z^h) \R u^h +
\sum_{e=1}^{n_{el}} \int_{\Omega} (\adva{z^h})(\uptau^e \R u^h)
\, \text{d} \Omega \\
&=
\int_{\Omega} (\tilde{z}' + \tilde{z}) \R u^h +
\sum_{e=1}^{n_{el}} \int_{\Omega} (\adva{z^h})(\uptau^e \R u^h)
\, \text{d} \Omega \\
&=
\int_{\Omega} \tilde{z}' \R u^h +
\sum_{e=1}^{n_{el}} \int_{\Omega} (\adva{z^h})(\uptau^e \R u^h)
\, \text{d} \Omega + \\
& \quad \quad \quad \int_{\Omega} \tilde{z} \R u^h \\
&=
\int_{\Omega} (\uptau^e_d \R^* z^h) \R u^h +
\sum_{e=1}^{n_{el}} \int_{\Omega} (\adva{z^h})(\uptau^e \R u^h)
\, \text{d} \Omega + \\
& \quad \quad \quad \int_{\Omega} \tilde{z} \R u^h
\end{aligned}
\end{gather*}
%
where the first equality is by definition \eqref{eq:functional},
the second equality is due to the dual PDE
\eqref{eq:app_dual_pde}, the third equality is due to the
fundamental relationship \eqref{eq:app_adjoint},
the fourth equality is due to the primal PDE
\eqref{eq:app_primal_pde}, the fifth equality is due to
the definition of the primal residual \eqref{eq:app_primal_residual},
the sixth equality is due to the primal subgrid model
\eqref{eq:app_primal_subgrid} (where we have added and
subtracted equal terms), the seventh equality is due to
linearity, the eighth equality is due to the decomposition
of the dual solution \eqref{eq:primal_subgrid_decomp},
the ninth equality is due to linearity, and the tenth equality
is due to the fine-scale approximation to the dual solution
\eqref{eq:dual_fine_approx}.

%=============================================================================
\subsection{Proposition 5}
%=============================================================================

We first note that
the derivation in \ref{app:adjoint} can
be carried out in exactly the same manner
for $u,z \in H^1_0(\Omega^e)$ to obtain the result:
%
\begin{gather}
\begin{aligned}
&\int_{\Omega^e} z(\adv{u}) \, \text{d} \Omega = \\
&\int_{\Omega^e} (\adva{z}) u \, \text{d} \Omega
\quad \forall \, u,z \in H^1_0(\Omega^e)
\end{aligned}
\label{eq:app_elem_adjoint}
\end{gather}
%
We note that the problem \eqref{eq:primal_elem_greens}
defining the primal element-level Green's function implies that
$g^e(\bs{x}; \bs{y}) \in H^1_0(\Omega^e)$.
Similarly, the dual element-level Green's function satisfies
$g^e_d(\bs{x}; \bs{y}) \in H^1_0(\Omega^e)$ from equation
\eqref{eq:dual_elem_greens}. With this information,
we utilize the relationship \eqref{eq:app_elem_adjoint} to verify
that $g^e(\bs{x}; \bs{y}) = g^e_d(\bs{x}; \bs{y})$, even though
the operator $\OP$ is not self-adjoint.
%
\begin{gather*}
\begin{aligned}
&\adva{ g^e(\bs{x}; \bs{y}) }
=
\delta(\bs{x} - \bs{y}) \\
\implies &\int_{\Omega^e}
g^e_d(\bs{x}; \bs{z}) (\adva{ g^e(\bs{x}; \bs{y}) })
\, \text{d} \Omega
= \\
& \quad \quad \int_{\Omega^e}
g^e_d(\bs{x}; \bs{z}) \delta(\bs{x} - \bs{y})
\, \text{d} \Omega \\
\implies
& \int_{\Omega^e} (\adv{ g^e_d(\bs{x}; \bs{z})}) g^e(\bs{x}; \bs{y})
\, \text{d} \Omega
= \\
& \quad \quad \int_{\Omega^e}
g^e_d(\bs{x}; \bs{z}) \delta(\bs{x} - \bs{y})
\, \text{d} \Omega \\
\implies
& \int_{\Omega^e} \delta(\bs{x} - \bs{z}) g^e(\bs{x}; \bs{y})
\, \text{d} \Omega
=
\int_{\Omega^e}
g^e_d(\bs{x}; \bs{z}) \delta(\bs{x} - \bs{y})
\, \text{d} \Omega \\
\implies
& g^e(\bs{z}; \bs{y})
= g^e_d(\bs{y}; \bs{z}),
\end{aligned}
\end{gather*}
%
Here the first equality is due to the definition of the
primal element-level Green's function \eqref{eq:primal_elem_greens},
the second equality is achieved by multiplying by the dual
element-level Green's function and integrating over the element domain,
the third equality is due to the fundamental relationship
\eqref{eq:app_elem_adjoint}, and the fourth equality is due to the
definition of the dual element-level Green's function
\eqref{eq:dual_elem_greens}.
