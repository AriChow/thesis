%%% CONCLUSIONS AND FUTURE WORK
\chapter{CONCLUSIONS AND FUTURE WORK}
\label{chap:conclusions}

%%% CONCLUSIONS
\section{Conclusions}

We have presented an automated approach for adjoint-based error estimation and mesh
adaptation to approximate and control the discretization error associated with
functional quantities. In scenarios when these functional quantities correspond
to physically meaningful outputs, such as the average von-Mises stress over a
sub-domain, we have demonstrated that this approach can provide effective error
estimates and meaningfully adapt the mesh to reduce the functional functional
discretization error. In particular, we have developed and implemented this approach
to be applicable to stabilized finite element methods and have demonstrated its
ability to effectively estimate and control errors in a variety of applications
in solid mechanics. Importantly, we have demonstrated that this approach is applicable
to applications in solid mechanics with complex three-dimensional geometries.
Further, we have demonstrated the ability of this approach
to execute effectively on parallel machines. Additionally, we have extended the
automated approach to investigate two novel strategies for solving adjoint problems
on non-uniformly refined nested meshes. Finally, we have developed and
investigated a novel approach for adjoint-based error estimation and mesh adaptation
in the context of variational multiscale finite element methods. For the purposes
of mesh adaptation, we have demonstrated the superiority of this approach when compared
to a previously developed error estimate.

%%% FUTURE WORK
\section{Future Work}

\subsection{Higher Order Finite Element Methods}

In this work, we have preferred to utilize low-order stabilized
finite element methods to provide numerical stability in solid
mechanics applications with incompressibility constraints.
Taylor-Hood type elements provide another approach to develop
stable discretizations in the finite element method, where
displacements are represented with basis functions of order
$p+1$ and pressures are represented with basis functions of
order $p$. This leads to a more involved finite element assembly
process, but the techniques used in the Goal application
could be extended to account for this to investigate adjoint-based
error estimation and mesh adaptation for solid mechanics with
higher-order finite element methods.

\subsection{Extending Capabilities to Quasi-Steady/Transient Problems}

The examples presented in this work have been either steady-state examples
or quasi-steady examples loaded in a single load step.
The Goal application currently has the ability to perform multiple load
steps and solve the adjoint problem either at the end of each load step
or at the end of the total number of load steps to estimate errors in functional
quantities of interest. This approach is mathematically valid for constitutive
models that lack history-dependent variables, such as neo-Hookean elasticity.
However, for constitutive models with history dependence, an error is
introduced by the choice of step size, even if the problem is not truly
transient. A natural extension of the current work is to consider multiple
quasi-steady load steps for constitutive models with history-dependent variables
and truly transient problems with inertial terms included in the balance
of linear momentum residual. In these scenarios, the mathematical analysis
requires the adjoint problem to be solved \emph{backwards in time}.
In parallel, the effective implementation of such a backwards in time adjoint
problem is an ongoing research topic in the CFD community.

\subsection{Extending VMS Techniques for Solid Mechanics}

We have developed and investigated enriching the adjoint solution with
variational multiscale (VMS) techniques and estimating errors in
linear functional QoIs in the context of a linear advection-diffusion model problem.
These techniques should be analysed in the context of nonlinear variational
problems and QoIs. In the Goal application, we have additionally included
the ability to solve the adjoint problem on the same finite element space
as used for the primal problem. This provides a convenient pretext to
implement VMS adjoint enrichment routines in the Goal application, and
investigate error estimates obtained as such.
