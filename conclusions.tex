%%% CONCLUSIONS AND FUTURE WORK
\chapter{CONCLUSIONS AND FUTURE WORK}
\label{chap:conclusions}
\section{Conclusions}
We investigate methods and algorithms to perform optimization and quantification of error in image classification pipelines. Our work revolves around the fact that the error observed as the classification metrics at the end of image classification pipelines is due to the accumulation and propagation of error from different components of the image classification pipeline. To this end, we present the work done on four different projects. Two of the projects are focused on minimization of error with respect to different perspectives in the image classification pipeline. 
The first project shows how the classification error can be reduced by modifying a particular step (image dataset pre-processing) of the image classification pipeline. In particular we propose a method to generate artificial data using a parametric 3D model of blood vessels. This is used as a data augmentation procedure to reduce the error in image classification caused due to class imbalance. Class imbalance is a common source of error in image classification problems in the medical domain. We show empirical results that this method of generating artificial 3D models of vasculature can characterize the morphology of blood vessels in neuropathological tissue samples.  
The second work involves the minimization of error from the data analysis steps in the pipeline as a whole using exhaustive grid search. This method finds the best configuration of algorithms and hyper-parameters that minimizes the cross-validation error for two classification tasks in the domain of material science. We also show that pre-trained convolutional neural networks maybe used  as a good feature representation for dendritic and non-dendritic microstructures. 
In the second part of the thesis, we tackle the problem of quantification of error from different parts of an image classification pipeline. A lot of work has been done in the minimization of error in image classification with respect to optimizing different components in the pipeline. However, little attention has been paid to understanding the origin of error from the components of the workflow. 
The first project in this part involves the quantification of the quality of an image using a machine learning based score. This score can automatically suggest to a domain expert whether an image should or should not be used for further analysis. This provides a way to filter the image datasets in scientific domains based on the quality of images for analysis.
The second project proposes a methodology to quantify the contribution of error in image classification pipelines using an \textit{agnostic} methodology. This method is able to quantify the contribution of errors from computational steps and algorithms in the pipeline. This provides a way for domain experts and data scientists to diagnose and profile an image classification pipeline and improve them based on the results. In addition, we also show that hyperparameter optimization techniques like random search can compute the contribution of errors similar to exhaustive grid search.

%%% CONCLUSIONS
\section{Future work}
There are various extensions that could be made in terms of future work.
\begin{enumerate}

\item The exhaustive grid search approach for characterizing microstructures in the domain of material science can be used as a standard technique for finding the best configuration of algorithms and hyper-parameters for an image classification problem in that domain. In addition, more research can be performed to improve upon the pre-trained convolutional neural networks that have been successfully used to characterize images of microstructures in material science.
\item The problem of data or class imbalance is rampant in the scientific domain and in particular in the medical domain. Therefore, the parametric 3D model approach to perform data augmentation can be used as a data balancing methodology in other scientific domains that involve the analysis of images. 
\item The \textit{Quality of Image} score described in the first project maybe improved by using more sophisticated techniques and algorithms to quantify the error. In addition, the method can also be used to test the efficacy of the approach if more data from the same dataset is available. The experiment would take the form of comparing the results of data analysis using the entirety of the unfiltered dataset and the filtered \textit{good} dataset using the score. The hypothesis then is that the results from the \textit{good} dataset is better because it consists of images that consist of high quality image samples.

\item The \textit{agnostic} methodology to quantify the error contribution in image classification pipelines in scientific domains can also be extended  to understand and interpret end-to-end models like convolutional neural networks. In that case, the error contributions could be computed with respect to the layers of the neural network and could help deep learning experts to diagnose the source of error in the models. In addition, this work may also be extended to other data domains and pipelines like machine learning workflows based on text and speech in natural language processing. This approach maybe used to quantify errors in any pipeline system that involves an optimization framework. 

\end{enumerate}