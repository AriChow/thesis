\chapter{FORWARD AUTOMATIC DIFFERENTIATION}
\label{chap:fad}

%%% INTRODUCTION
\section{Introduction}

Automatic differentiation (AD) is a useful technique to computationally
evaluate analytic derivatives (to machine precision) of a given function.
Numerical software is necessarily executed as a sequence of elementary
operations, and AD operates by applying the chain rule to this sequence
of elementary operations. There are two standard modes of AD, forward and
reverse, as well as two methods of implementing the technique, operator
overloading and source code transformation. Presently, we consider the
forward mode of automatic differentiation using operator overloading in the
context of the \texttt{C++} programming language. For an excellent
comprehensive overview of automatic differentiation, see
\cite{griewank2008evaluating}.

%%% FAD WITH OPERATOR OVERLOADING
\section{Forward AD with Operator Overloading}

\texttt{C++} provides the capability to overload standard operators such as
$+,-,*,$ and $/$ for custom data types. In scientific computing, operator
overloading is convenient in that it allows developers the ability to program
using a notation much closer the the target mathematical notation. A common
example in scientific computing is the use of operator overloading to implement
matrix multiplication. Say we have defined a \texttt{C++} dense matrix
container called MyMatrix. Using operator overloading, a developer can define
matrix multiplication specific behavior for the $*$ operator for the MyMatrix
variable. Listing \ref{lst:opover} demonstrates the convenience of using
operator overloading to multiply two MyMatrix objects.
%
\begin{lstlisting}[
style=mystyle,
caption=Using operator overloading for a custom matrix class,
label=lst:opover]
MyMatrix A = ...
MyMatrix B = ...
MyMatrix C = A*B;
\end{lstlisting}
%

In the forward mode AD, a \texttt{C++} class variable container is defined
that stores both a scalar value $x$ that corresponds the value of the variable
and a derivative array $\bs{x}'$ that corresponds the values of derivatives of
the variable with repsect to chosen independent variables. The computer
program is then written in terms of these AD variables. At the beginning of
the program, appropriate AD variables are initialized to their appropriate
values, and \emph{seeded}. Seeding refers to appropriately setting the
derivative array of the AD variables. For example, the $i^th$ variable in a
system of $n$ independent variables could be represented as:
%
\begin{gather}
x_i \quad
[ \frac{\partial x_i}{\partial x_1},
\frac{\partial x_i}{\partial x_2},
\dots, \frac{\partial x_i}{\partial x_n} ],
\end{gather}
%
where $x_i$ is the variable value and the terms contained in the brackets
represent the AD variable's derivative array. To appropriately seed the
derivative array, all values then become zero except for the $i^th$ component,
which becomes 1.
%
\begin{gather}
x_i \quad [0, 0, \dots, 1, \dots, 0, 0].
\end{gather}

Over the course of the execution of the software program, intermediate
derivative arrays are updated via operator overloading when a new operator is
encouterd. The operators are overloaded with definitions from basic calclulus
rules. For example, the result for the derivative array of an AD variable $a$
times another AD variable $b$ using the overloaded $*$ operator would result
in
%
\begin{gather}
a \cdot b \quad [ \bs{a}' \cdot  b + a \cdot \bs{b}' ],
\end{gather}
%
where the resultant value of the AD variable is simply the multiplied value
$a \cdot b$, but the derivative array is updated according to the product rule.
Here $\bs{a}' \cdot b$ is the scalar value of the AD variable $b$ times $a$'s
derivative array and $a \cdot \bs{b}'$ is the scalar value of $a$ times
$b$'s derivative array.

\begin{figure}
\centering
\begin{tabular}{|  l | l |}
\hline
Operation & Derivative rule \\ \hline
$c = a \pm b$ & $\bs{c}' = \bs{a}' \pm \bs{b}'$ \\ \hline
$c = ab$      & $\bs{c}' = a \bs{b}' + b \bs{a}'$ \\ \hline
$c = a/b$     & $\bs{c}' = (\bs{a}' b - \bs{b}' a) / b^2$ \\ \hline
$c = \sin(a)$  & $\bs{c}' = \cos(a) \bs{a}'$ \\ \hline
$c = \cos(a)$  & $\bs{c}' = -\sin(a) \bs{a}'$ \\ \hline
$c = \exp(a)$  & $\bs{c}' = \exp(a) \bs{a}'$ \\ \hline
$c = \log(a)$  & $\bs{c}' = \bs{a}'/a $ \\ \hline
\end{tabular}
\end{figure}

Now, if we program the computation of a given function
$f(x_i)$, $i=1,2,\dots,n$, with AD variables, we necessarily obtain both the
value of the function $f$ and its gradient $\nabla f = \bs{f}'$. This is by
virtue of the fact that derivatives have been propogated forward through the
code from the \emph{seed} point to the evaluation of $f$.

From an implementation point of view, automatic differentiation via operator
overloading is attractive in that existing codes can be easily modified to
obtain gradient information. For example, consider a code that uses doubles
for all of its evaluation types. A developer could simply replace appropriate
instances of `double` in the code with the appropriate AD type name. Through
very little development cost, gradient information is obtained.

%%% A SIMPLE EXAMPLE
\section{A Simple Example}

To further illustrate the concept of forward AD, we present a simple example
using Sandia's Sacado automatic differentiation library and examine it in
depth. Listing \ref{lst:sacadoex} presents a simple example of the
computation of three functions using forward automatic differentiation and
displays the resulting values and gradients of the evaluated functions.

\begin{lstlisting}[
style=mystyle,
caption=A simple forward AD example,
label=lst:sacadoex]
#include <Sacado.hpp>
typedef Sacado::Fad::DFad<double> FAD;
FAD f(FAD x, FAD y) { return x + y; }
FAD g(FAD x, FAD y) { return x * y; }
int main() {
  FAD x = 2.0;
  FAD y = 3.0;
  x.diff(0,2);
  y.diff(1,2);
  std::cout << x << std::endl;
  std::cout << y << std::endl;
  std::cout << f(x,y) << std::endl;
  std::cout << g(x,y) << std::endl;
  std::cout << f(x,y) * g(x,y) << std::endl;
}
\end{lstlisting}

Line 1 includes the Sacado header, which is an automatic differentiation
library developed by Sandia National Laboratories. Line 2 declares the
forward AD type to be called FAD. Line 3 declares $f$ to be the sum of two
AD variables $x$ and $y$. Line $4$ declares $g$ to be the product of two AD
variables $x$ and $y$. Line 10 initializes the AD variable $x$ to a value of
$2$ and line 12 \emph{seeds} $x$ to be the 1st variable in a system of $2$
independent variables. Line 11 initializes the AD variable $y$ to a value of
$3$ and line 13 \emph{seeds} $y$ to be the 2nd variable in a system of $2$
independent variables. Lines 14-18 print the results of evaluating various
functions.

The output from this simple example program is shown below
%
\begin{verbatim}
                    x:      2 [ 1 0 ]
                    y:      3 [ 0 1 ]
                    f:      5 [ 1 1 ]
                    g:      6 [ 3 2 ]
                    f*g:    30 [ 21 16 ]
\end{verbatim}
%
Naturally, the value of $x=2$ and $y=3$, as we initialized them to be.
Similarly, the derivative array of $x$ shows that
$\frac{\partial x}{\partial x} = 1$ and $\frac{\partial x}{\partial y} = 0$
and the derivative array of $y$ shows that $\frac{\partial y}{\partial x} = 0$
and $\frac{\partial y}{\partial y} = 1$.

The value of $f(x,y) = z+ y$ is given as $2+3=5$, and its derivative array is
updated according to operator overloading of the $+$ operator as the sum of
$x$'s derivative array and $y$'s derivative array:
%
\begin{gather}
\bs{f}' = \bs{x'} + \bs{y}' = [1 \; 0] + [0 \; 1] = [1 \; 1].
\end{gather}

The value of $g(x,y) = x \cdot y$ is given as $2\cdot3 = 6$, and its
derivative array is updated according to operator overloading of the $*$
operator as the sum of $\bs{x}'$ times $y$ and $\bs{y}'$ times $x$:
%
\begin{gather}
\bs{g}' = \bs{x}' \cdot y + x \cdot \bs{y}' =
[1 \; 0] \cdot 3 + 2 \cdot [0 \; 1] =
[3 \; 2].
\end{gather}

Finally, the value of $h(x,y) = f(x,y) \cdot g(x,y)$ is computed as $6*5 = 30$,
and its derivative array is computed by propogating $\bs{f}'$ and $\bs{g}'$
through the code as:
%
\begin{gather}
\bs{h}' = \bs{f}' \cdot g + f \cdot \bs{g}' =
[1 \; 1] \cdot 6 + 5 \cdot [3 \; 2] =
[21 \; 16].
\end{gather}
